\section[Kapitel�berschrift im Inhaltsverzeichnis]{Viel l�ngere Kapitel�berschrift, die auch erst auf der Seite und nicht im Verzeichnis erscheint}\label{sec:motivation}
Ich\marginline{Pssst!} bin Blindtext. Von Geburt an. Es hat lange gedauert, bis ich begriffen habe, was es bedeutet, ein blinder Text zu sein: Man macht keinen Sinn. Man wirkt hier und da aus dem Zusammenhang gerissen. Oft wird man gar nicht erst gelesen. Aber bin ich deshalb ein schlechter Text? Ich wei�, dass ich nie die Chance haben werde im Stern zu erscheinen. Aber bin ich darum weniger wichtig? Ich bin blind! Aber ich bin gerne Text. Und sollten Sie mich jetzt tats�chlich zu Ende lesen, dann habe ich etwas geschafft, was den meisten "normalen" Texten nicht gelingt. Ich bin Blindtext. Von Geburt an. Es hat lange gedauert, bis ich begriffen habe, was es bedeutet, ein blinder Text zu sein: Man macht keinen Sinn. Man wirkt hier und da aus dem Zusammenhang gerissen. Oft wird man gar nicht erst gelesen. Aber bin ich deshalb ein schlechter Text? Ich wei�, dass ich nie die Chance haben werde im Stern zu erscheinen. Aber bin ich darum weniger wichtig?

Eine\marginline{Hey, du!} wunderbare Heiterkeit hat meine ganze Seele eingenommen, gleich den s��en Fr�hlingsmorgen, die ich mit ganzem Herzen genie�e. Ich bin allein und freue mich meines Lebens in dieser Gegend, die f�r solche Seelen geschaffen ist wie die meine. Ich bin so gl�cklich, mein Bester, so ganz in dem Gef�hle von ruhigem Dasein versunken, da� meine Kunst darunter leidet. Ich k�nnte jetzt nicht zeichnen, nicht einen Strich, und bin nie ein gr��erer Maler gewesen als in diesen Augenblicken. Wenn das liebe Tal um mich dampft, und die hohe Sonne an der Oberfl�che der undurchdringlichen Finsternis meines Waldes ruht, und nur einzelne Strahlen sich in das innere Heiligtum stehlen, ich dann im hohen Grase am fallenden Bache liege, und n�her an der Erde tausend mannigfaltige Gr�schen mir merkw�rdig werden; wenn ich das Wimmeln der kleinen Welt zwischen Halmen, die unz�hligen, unergr�ndlichen Gestalten der W�rmchen, der M�ckchen n�her an meinem Herzen f�hle, und f�hle die Gegenwart des Allm�chtigen, der uns nach seinem Bilde schuf, das Wehen des Alliebenden, der uns in ewiger Wonne schwebend tr�gt und erh�lt; mein Freund! 