%	Dies ist eine Vorlage von Tobias Banaszak
%	tobias.banaszak@gmail.com
%	Sie wurde 2010 erstellt und mehrfach ver�ndert und angepasst.
%	Zum kompilieren wurde pdfLaTeX benutzt. 
%	Einen Umschlag/Einband liefert diese Vorlage nicht, da die Schriftart der FH Aachen nicht mal so eben in *TeX genutzt werden kann und es genau daf�r in diesem Fall schon eine Indesign-Vorlage gibt
%
\documentclass[
12pt,								% Schriftgre (12pt, 11pt (Standard))
oneside,							% einseitiges Layout
listof=flat,						% setzt z.B. das Abbvz neu, falls die Nummern in die Beschreibung rein ragt
%draft,								% �berlange Zeilen in Ausgabe kennzeichnen, setzt keine Bilder un Listings.
numbers=noenddot,		% macht aus der �berschrift 1.1. den letzten Punkt weg
%abstracton						%setzt "abstact"/"zusammenfassung"
]{scrartcl}
%
%%%%%%%%%%%%%%%%%%%%%%%%%%%%%%%%%%%%%%
% Pakete einbinden etc...
\input{pre.tex}
%%%%%%%%%%%%%%%%%%%%%%%%%%%%%%%%%%%%%%
%f�r den PDF-Reader
\hypersetup{
	pdfpagelayout=TwoPageRight,	%erste Seite ist Titelseite
	pdfstartview=Fit,
	pdfauthor={Vorname Nachname}, 
	pdftitle={Titel Titel Titel},
	pdfsubject={Masterarbeit},
	pdfkeywords={Masterarbeit | Stichwort| Stichw�rter | blabla}
}
%%%%%%%%%%%%%%%%%%%%%%%%%%%%%%%%%%%%%%
%
%
\begin{document}
%	Standartumgebung f�r Codelistings setzen
\lstset{showstringspaces=false,frame=lines,breaklines=true,language=[Sharp]C,numberbychapter=true,captionpos=b,prebreak={\Righttorque},basicstyle=\footnotesize}
%
% Seitengeometrie f�r den Titelseite & Verzeichnisse anpassen
\newgeometry{left=2.7cm,right=2.7cm,top=3.05cm,bottom=3.3cm,marginparsep=0mm,marginparwidth=0.0cm}
%%%%%%%%%%%%%%%%%%%%%%%%%%%%%%%%%%%%%%
%%%%%%%%%%%%%%%%%%%%%%%%%%%%%%%%%%%%%%
%	Titelseite
\begin{titlepage}%
\thispagestyle{empty}%
\singlespace
\vspace*{6cm}
\huge Masterarbeit\newline%\newline
%\normalsize zur Erreichung des akademischen Grades \\
%\normalsize Master of Engineering\vspace*{0.5cm}
%
%\Huge \textcolor{mint}{\textbf{Konzeption und Entwicklung\newline einer multi-modalen\newline Connected-Mobility-App}}\vspace{2cm}
\huge\textcolor{mint}{Ganz viel Platz f�r bahnbrechende Titel von bahnbrechenden Arbeiten}\newline
\huge Vorname Nachname\vspace{2cm}

\normalsize\textbf{Hochschule Ort\\
Fachbereich f�r Dinge}\\
{\color{mint}Studiengang}\\

\textbf{Vorgelegt von}\\
Hans Wurst, B. Sc.\\
Matrikelnummer: 123456\\
Aachen, \today\\

\textbf{Pr�fer}\\
 Prof. Dr.-Ing. Rainer Wahnsinn\\
Max Power, M. Eng.\\
%%
\end{titlepage}%
\onehalfspacing
\newpage
%%%%%%%%%%%%%%%%%%%%%%%%%%%%%%%%%%%%%%
%	Erzeugen von Verzeichnissen
\renewcommand{\listfigurename}{Abbildungsverzeichnis}
\markright{}%gegen die zweite zeile im header
\tableofcontents			% Inhaltsverzeichnis
\newpage
%
\markright{}					% gegen die zweite zeile im header
%\listoftables				% Tabellenverzeichnis
\listoffigures				% Abbildungsverzeichnis
\newpage
%\thispagestyle{empty}
\section*{Abk�rzungsverzeichnis}
\sectionmark{Abk�rzungsverzeichnis}
In der Reihenfolge ihres Auftretens:
\begin{acronym}[Abk�rzungen]
	\acro{Abk.}{Abk�rzung}
	\acro{z.B.}{Zum Beispiel}
\end{acronym}
\newpage
%
%%%%%%%%%%%%%%%%%%%%%%%%%%%%%%%%%%%%%%
%%%%%%%%%%%%%%%%%%%%%%%%%%%%%%%%%%%%%%
%	Der Text
\restoregeometry			% Seitengeometrie zur�ck setzen
%\blinddocument
%
%	Motivation
\section[Kapitel�berschrift im Inhaltsverzeichnis]{Viel l�ngere Kapitel�berschrift, die auch erst auf der Seite und nicht im Verzeichnis erscheint}\label{sec:motivation}
Ich\marginline{Pssst!} bin Blindtext. Von Geburt an. Es hat lange gedauert, bis ich begriffen habe, was es bedeutet, ein blinder Text zu sein: Man macht keinen Sinn. Man wirkt hier und da aus dem Zusammenhang gerissen. Oft wird man gar nicht erst gelesen. Aber bin ich deshalb ein schlechter Text? Ich wei�, dass ich nie die Chance haben werde im Stern zu erscheinen. Aber bin ich darum weniger wichtig? Ich bin blind! Aber ich bin gerne Text. Und sollten Sie mich jetzt tats�chlich zu Ende lesen, dann habe ich etwas geschafft, was den meisten "normalen" Texten nicht gelingt. Ich bin Blindtext. Von Geburt an. Es hat lange gedauert, bis ich begriffen habe, was es bedeutet, ein blinder Text zu sein: Man macht keinen Sinn. Man wirkt hier und da aus dem Zusammenhang gerissen. Oft wird man gar nicht erst gelesen. Aber bin ich deshalb ein schlechter Text? Ich wei�, dass ich nie die Chance haben werde im Stern zu erscheinen. Aber bin ich darum weniger wichtig?

Eine\marginline{Hey, du!} wunderbare Heiterkeit hat meine ganze Seele eingenommen, gleich den s��en Fr�hlingsmorgen, die ich mit ganzem Herzen genie�e. Ich bin allein und freue mich meines Lebens in dieser Gegend, die f�r solche Seelen geschaffen ist wie die meine. Ich bin so gl�cklich, mein Bester, so ganz in dem Gef�hle von ruhigem Dasein versunken, da� meine Kunst darunter leidet. Ich k�nnte jetzt nicht zeichnen, nicht einen Strich, und bin nie ein gr��erer Maler gewesen als in diesen Augenblicken. Wenn das liebe Tal um mich dampft, und die hohe Sonne an der Oberfl�che der undurchdringlichen Finsternis meines Waldes ruht, und nur einzelne Strahlen sich in das innere Heiligtum stehlen, ich dann im hohen Grase am fallenden Bache liege, und n�her an der Erde tausend mannigfaltige Gr�schen mir merkw�rdig werden; wenn ich das Wimmeln der kleinen Welt zwischen Halmen, die unz�hligen, unergr�ndlichen Gestalten der W�rmchen, der M�ckchen n�her an meinem Herzen f�hle, und f�hle die Gegenwart des Allm�chtigen, der uns nach seinem Bilde schuf, das Wehen des Alliebenden, der uns in ewiger Wonne schwebend tr�gt und erh�lt; mein Freund! 
%
%	Stand der Technik
\section{Einleitung}\label{chap:standdertechnik}
Es\marginline{Jaaa, genau du!} gibt im Moment in diese Mannschaft, oh, einige Spieler vergessen ihnen Profi was sie sind. Ich lese nicht sehr viele Zeitungen, aber ich habe geh�rt viele Situationen. Erstens: wir haben nicht offensiv gespielt. Es gibt keine deutsche Mannschaft spielt offensiv und die Name offensiv wie Bayern. Letzte Spiel hatten wir in Platz drei Spitzen: Elber, Jancka und dann Zickler. Wir m�ssen nicht vergessen Zickler. Zickler ist eine Spitzen mehr, Mehmet eh mehr Basler. Ist klar diese W�rter, ist m�glich verstehen, was ich hab gesagt? Danke. Offensiv, offensiv ist wie machen wir in Platz. Zweitens: ich habe erkl�rt mit diese zwei Spieler: nach Dortmund\footnote{Dies ist ein Test. Ich wei� n�mlich gerade gar nicht, wie die Fu�noten hier aussehen.} brauchen vielleicht Halbzeit Pause. Ich habe auch andere Mannschaften gesehen in Europa nach diese Mittwoch. Ich habe gesehen auch zwei Tage die Training. Ein Trainer ist nicht ein Idiot! Ein Trainer sei sehen was passieren in Platz. In diese Spiel es waren zwei, drei diese Spieler waren schwach wie eine Flasche leer! Haben Sie gesehen Mittwoch, welche Mannschaft hat gespielt Mittwoch? Hat gespielt Mehmet oder gespielt Basler oder hat gespielt Trapattoni? Diese Spieler beklagen mehr als sie spielen! Wissen Sie, warum die Italienmannschaften kaufen nicht diese Spieler? Weil wir haben gesehen viele Male solche Spiel!
\subsection{Er h�rte leise...}
Er\marginline{Pssst...} h�rte leise Schritte hinter sich. Das bedeutete nichts Gutes. Wer w�rde ihm schon folgen, sp�t in der Nacht und dazu noch in dieser engen Gasse mitten im �bel beleumundeten Hafenviertel? Gerade jetzt, wo er das Ding seines Lebens gedreht hatte und mit der Beute verschwinden wollte! Hatte einer seiner zahllosen Kollegen dieselbe Idee gehabt, ihn beobachtet und abgewartet, um ihn nun um die Fr�chte seiner Arbeit zu erleichtern?

Dies\marginline{Willst du...} ist ein Typoblindtext. An ihm kann man sehen, ob alle Buchstaben da sind und wie sie aussehen. Manchmal benutzt man Worte wie Hamburgefonts, Rafgenduks oder Handgloves, um Schriften zu testen. Manchmal S�tze, die alle Buchstaben des Alphabets enthalten - man nennt diese S�tze �Pangrams�. Sehr bekannt ist dieser: The quick brown fox jumps over the lazy old dog. Oft werden in Typoblindtexte auch fremdsprachige Satzteile 

In\marginline{ein X kaufen?} Lateinisch sieht zum Beispiel fast jede Schrift gut aus. Quod erat demonstrandum. Seit 1975 fehlen in den meisten Testtexten die Zahlen, weswegen nach TypoGb. 204 � ab dem Jahr 2034 Zahlen in 86 der Texte zur Pflicht werden. Nichteinhaltung wird mit bis zu 245 ? oder 368 \$ bestraft. Genauso wichtig in sind mittlerweile auch ��c��t�, die in neueren Schriften aber fast immer enthalten sind. Ein wichtiges aber schwierig zu integrierendes Feld sind OpenType-Funktionalit�ten. 

\subsection{Ich bin blind!}
\blindtext
\subsubsection{Eieieiei}
\blindtext
%
%	Fazit und Ausblick
\section{Fazit und Ausblick}\label{sec:Ausblick}%
Zwei\marginline{Gehen Sie bitte weiter!} flinke Boxer jagen die quirlige Eva und ihren Mops durch Sylt. Franz jagt im komplett verwahrlosten Taxi quer durch Bayern. Zw�lf Boxk�mpfer jagen Viktor quer �ber den gro�en Sylter Deich. Vogel Quax zwickt Johnys Pferd Bim. Sylvia wagt quick den Jux bei Pforzheim. Polyfon zwitschernd a�en M�xchens V�gel R�ben, Joghurt und Quark.

"Fix, Schwyz!"\marginline{Hier gibt es nichts zu sehen!} qu�kt J�rgen bl�d vom Pa�. Victor jagt zw�lf Boxk�mpfer quer �ber den gro�en Sylter Deich. Falsches �ben von Xylophonmusik qu�lt jeden gr��eren Zwerg. Heiz\-�l\-r�ck\-sto�\-ab\-d�mpf\-ung. Zwei flinke Boxer jagen die quirlige Eva und ihren Mops durch Sylt. Franz jagt im komplett verwahrlosten Taxi quer durch Bayern.
\begin{figure}[!h]
	\centering
	\includegraphics[width=0.5\linewidth]{Bilder/Testbild.png}
	\caption{Testbild}
	\label{fig:testbild}
\end{figure}

Zw�lf\marginline{Bilder?} Boxk�mpfer jagen Viktor quer �ber den gro�en Sylter Deich. Vogel Quax zwickt Johnys Pferd Bim. Sylvia wagt quick den Jux bei Pforzheim. Polyfon zwitschernd a�en M�xchens V�gel R�ben, Joghurt und Quark. "Fix, Schwyz! " qu�kt J�rgen bl�d vom Pa�. Victor jagt zw�lf Boxk�mpfer quer �ber den gro�en Sylter Deich.

\subsection{Untersektion}\label{ssec:Untersektion}
Im\marginline{Referenzen?} Text kann man auf Bilder referenzieren. In Abb. \ref{fig:testbild} auf Seite \pageref{fig:testbild} sieht man z.B. ein Testbild. Dabei muss in der \texttt{figure} immer das \texttt{label} nach der \texttt{caption} sein. Sonst sind die Referenzen falsch! Man kann nat�rlich auch auf die Quellen referenzieren. In \cite[S. 12]{dasisteinemarke} steht aber leider gar nichts drin.

\subsubsection{Unter der Untersektion}\label{sssec:UnterUntersektion}
Falsches\marginline{bla bla bla} �ben von Xylophonmusik qu�lt jeden gr��eren Zwerg. Heiz�lr�cksto�abd�mpfung. Zwei flinke Boxer jagen die quirlige Eva und ihren Mops durch Sylt. Franz jagt im komplett verwahrlosten Taxi quer durch Bayern.%
%
%
%%%%%%%%%%%%%%%%%%%%%%%%%%%%%%%%%%%%%%
%%%%%%%%%%%%%%%%%%%%%%%%%%%%%%%%%%%%%%
%	Der Anhang
\appendix							% Beginn des Anhangs
%geometrie wechseln
\newgeometry{left=2.7cm,right=2.7cm,top=3.05cm,bottom=3.3cm,marginparsep=0mm,marginparwidth=0.0cm}
\section*{Anhang}
\markboth{Anhang}{}			% Header aufh�bschen
%
%
\section{Quellcode-Ausz�ge}\label{app:sourcecode}
\begin{minipage}{\textwidth}
\begin{lstlisting}[caption={Verifizierung der IP-Adresse mit einem regul�ren Ausdruck},label={list:regex}]
Regex reg = new Regex(@"^149.201.([1-9]|[1-9][0-9]|1[0-9][0-9]|2[0-4][0-9]| 25[0-5]).([1-9]|[1-9][0-9]|1[0-9][0-9]|2[0-4][0-9]|25[0-5])\$");
if (!reg.IsMatch(TextBox_IP.Text)) {
	Fehlermeldung im Formular ausgeben
} else {
	Session["IP"] =  TextBox_IP.Text;
	//Starte hier die Suche
}
\end{lstlisting}
\end{minipage}
%
%
\section{Quellcode und Dokumentation}
Dieser Bachelorarbeit liegt eine CD bei. Da sind gro�artige Dinge drauf.
%
%
%	Lieraturverz.
\markboth{Literatur}{}
\markright{}
\bibliographystyle{alpha1}
\markboth{Literatur}{}
\bibliography{BA_Bib}
%
%
\pagebreak
%Erkl�rung
\input{Erklaerung.tex}
\thispagestyle{empty}			%keine Seitenzahlen
\end{document}