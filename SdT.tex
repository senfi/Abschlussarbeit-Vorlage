\section{Einleitung}\label{chap:standdertechnik}
Es\marginline{Jaaa, genau du!} gibt im Moment in diese Mannschaft, oh, einige Spieler vergessen ihnen Profi was sie sind. Ich lese nicht sehr viele Zeitungen, aber ich habe geh�rt viele Situationen. Erstens: wir haben nicht offensiv gespielt. Es gibt keine deutsche Mannschaft spielt offensiv und die Name offensiv wie Bayern. Letzte Spiel hatten wir in Platz drei Spitzen: Elber, Jancka und dann Zickler. Wir m�ssen nicht vergessen Zickler. Zickler ist eine Spitzen mehr, Mehmet eh mehr Basler. Ist klar diese W�rter, ist m�glich verstehen, was ich hab gesagt? Danke. Offensiv, offensiv ist wie machen wir in Platz. Zweitens: ich habe erkl�rt mit diese zwei Spieler: nach Dortmund\footnote{Dies ist ein Test. Ich wei� n�mlich gerade gar nicht, wie die Fu�noten hier aussehen.} brauchen vielleicht Halbzeit Pause. Ich habe auch andere Mannschaften gesehen in Europa nach diese Mittwoch. Ich habe gesehen auch zwei Tage die Training. Ein Trainer ist nicht ein Idiot! Ein Trainer sei sehen was passieren in Platz. In diese Spiel es waren zwei, drei diese Spieler waren schwach wie eine Flasche leer! Haben Sie gesehen Mittwoch, welche Mannschaft hat gespielt Mittwoch? Hat gespielt Mehmet oder gespielt Basler oder hat gespielt Trapattoni? Diese Spieler beklagen mehr als sie spielen! Wissen Sie, warum die Italienmannschaften kaufen nicht diese Spieler? Weil wir haben gesehen viele Male solche Spiel!
\subsection{Er h�rte leise...}
Er\marginline{Pssst...} h�rte leise Schritte hinter sich. Das bedeutete nichts Gutes. Wer w�rde ihm schon folgen, sp�t in der Nacht und dazu noch in dieser engen Gasse mitten im �bel beleumundeten Hafenviertel? Gerade jetzt, wo er das Ding seines Lebens gedreht hatte und mit der Beute verschwinden wollte! Hatte einer seiner zahllosen Kollegen dieselbe Idee gehabt, ihn beobachtet und abgewartet, um ihn nun um die Fr�chte seiner Arbeit zu erleichtern?

Dies\marginline{Willst du...} ist ein Typoblindtext. An ihm kann man sehen, ob alle Buchstaben da sind und wie sie aussehen. Manchmal benutzt man Worte wie Hamburgefonts, Rafgenduks oder Handgloves, um Schriften zu testen. Manchmal S�tze, die alle Buchstaben des Alphabets enthalten - man nennt diese S�tze �Pangrams�. Sehr bekannt ist dieser: The quick brown fox jumps over the lazy old dog. Oft werden in Typoblindtexte auch fremdsprachige Satzteile 

In\marginline{ein X kaufen?} Lateinisch sieht zum Beispiel fast jede Schrift gut aus. Quod erat demonstrandum. Seit 1975 fehlen in den meisten Testtexten die Zahlen, weswegen nach TypoGb. 204 � ab dem Jahr 2034 Zahlen in 86 der Texte zur Pflicht werden. Nichteinhaltung wird mit bis zu 245 ? oder 368 \$ bestraft. Genauso wichtig in sind mittlerweile auch ��c��t�, die in neueren Schriften aber fast immer enthalten sind. Ein wichtiges aber schwierig zu integrierendes Feld sind OpenType-Funktionalit�ten. 

\subsection{Ich bin blind!}
\blindtext
\subsubsection{Eieieiei}
\blindtext